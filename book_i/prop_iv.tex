	\pagestyle{euclidthm}
    \begin{figure}[h]
      \begin{subfigure}{0.35\textwidth}
        \begin{tikzpicture}[scale=0.8]
            \coordinate (C) at (0,0);
            \coordinate (B) at (4,1);
            \coordinate (A) at (3,4);
            \tikzsectorabc[fill=red]{(A)}{(B)}{(C)}{0.2}
            \tikzsectorabc[fill=blue]{(B)}{(C)}{(A)}{0.2}
            \tikzsectorabc[fill=yellow]{(C)}{(A)}{(B)}{0.2}
            \tikztriangle[blue][black][red]{(A)}{(B)}{(C)}
        \end{tikzpicture}
      \end{subfigure}
      \begin{subfigure}{0.35\textwidth}
        \begin{tikzpicture}[scale=0.8]
            \coordinate (C) at (0,0);
            \coordinate (B) at (4,1);
            \coordinate (A) at (3,4);
            \tikzsectorabc[fill=red]{(A)}{(B)}{(C)}{0.2}
            \tikzsectorabc[fill=blue]{(B)}{(C)}{(A)}{0.2}
            \tikzsectorabc[fill=yellow]{(C)}{(A)}{(B)}{0.2}
            \tikztriangle[blue][black][red]{(A)}{(B)}{(C)}
        \end{tikzpicture}
      \end{subfigure}
    \end{figure}

    \renewcommand{\cab}{
        \begin{tikzpicture}[scale=0.8]
            \coordinate (C) at (0,0);
            \coordinate (B) at (4,1);
            \coordinate (A) at (3,4);
            \tikzsectorabc[fill=yellow]{(C)}{(A)}{(B)}{0.2}
        \end{tikzpicture}
    }
    \renewcommand{\abc}{
        \begin{tikzpicture}[scale=0.8]
            \coordinate (C) at (0,0);
            \coordinate (B) at (4,1);
            \coordinate (A) at (3,4);
            \tikzsectorabc[fill=red]{(A)}{(B)}{(C)}{0.2}
        \end{tikzpicture}
    }
    \renewcommand{\bca}{
        \begin{tikzpicture}[scale=0.8]
            \coordinate (C) at (0,0);
            \coordinate (B) at (4,1);
            \coordinate (A) at (3,4);
            \tikzsectorabc[fill=blue]{(B)}{(C)}{(A)}{0.2}
        \end{tikzpicture}
    }
    \newcommand{\ab}{\tikzhline[red]{0.5cm}}
    \newcommand{\bc}{\tikzhline[blue]{0.5cm}}
    \newcommand{\ac}{\tikzhline[black]{0.5cm}}
    \begin{prop}{\lettrine[lines=2]{I}f}
      two triangles have two sides of the one respectively equal to two 
      sides of the other, (\ac to \ac and \ab to \ab) and the angles (\cab 
      and \cab) contained by those equals sides also equal; then their 
      bases or their sides (\bc and \bc) are also equal: and the remaining 
      and their remaining angles opposite to equal sides are respectively 
      equal ($\bca \equals \bca$ and  $\abc \equals \abc$): and the triangles are equal in 
      every respect.
  \end{prop}
	\begin{proof}
    Let the two triangles be conceived, to be so placed, that the vertex of the one of the equal angles, \cab or \cab ; shall fall upon that of the other, and \ac to coincide with \ac, then will \ac coincide with \ac is applied: consequently \bc will coincide with \bc, or two straight lines will enclose a space, which is impossible (\refax{10}), therefore $\bc = \bc$, 
	\end{proof}
