	\pagestyle{euclidthm}

    \newcommand{\coordspec}{
      \coordinate (A) at (0,0)    ;
      \coordinate (B) at (2,3.5);
      \coordinate (C) at (4,0) ;
      \path (A) -- (B) node[coordinate,pos=0.7] (B') {B'};
      \pgfresetboundingbox
    }

%%%%%%%%%%

%   \newcommand{\coordspec}{
%     \coordinate (A) at (0,5)    ;
%     \coordinate (B) at (-3.5,-2);
%     \coordinate (C) at (3.5,-2) ;
%     \path (A) -- (B) node[coordinate,pos=0.6] (D) {D};
%     \path (A) -- (C) node[coordinate,pos=0.6] (E) {E};
%     \pgfresetboundingbox
%   }
%   \begin{figure}[h]
%       \begin{tikzpicture}[scale=0.8]
%           \coordspec
%           \tikzsectorabc[fill=black]{(B)}{(A)}{(C)}{1cm}
%           \tikzsectorabc[fill=blue]{(E)}{(D)}{(A)}{1cm}
%           \tikzsectorabc[fill=blue]{(A)}{(E)}{(D)}{1cm}
%           \tikzsectorabc[fill=yellow]{(C)}{(D)}{(E)}{1cm}
%           \tikzsectorabc[fill=yellow]{(D)}{(E)}{(B)}{1cm}
%           \tikzsectorabc[fill=red]{(E)}{(B)}{(A)}{1cm}
%           \tikzsectorabc[fill=red]{(A)}{(C)}{(D)}{1cm}
%           \tikztriangle[yellow][blue][red]{(A)}{(B)}{(E)}
%           \tikztriangle[yellow][blue][red]{(A)}{(C)}{(D)}
%           \tikztriangle[red][black][red]{(A)}{(D)}{(E)}
%       \end{tikzpicture}
%   \end{figure}
%   \newcommand{\ADE}{
%       \begin{tikzpicture}[scale=0.1]
%           \coordspec
%           \tikztriangle[red][black][red]{(A)}{(D)}{(E)}
%           \pgfresetboundingbox
%           \path (A) -- (D) -- (E) -- cycle;
%       \end{tikzpicture}
%   }
%   \newcommand{\ABE}{
%       \begin{tikzpicture}[scale=0.2,baseline=-0.5cm]
%           \coordspec
%           \tikzsectorabc[fill=black]{(B)}{(A)}{(C)}{0.3cm}
%           \tikztriangle[yellow][blue][red]{(A)}{(B)}{(E)}
%           \pgfresetboundingbox
%           \path (A) -- (B) -- (E) -- cycle;
%            \draw[red,ultra thick] (E) -- (A) -- (D);
%       \end{tikzpicture}
%   }
%   \newcommand{\ACD}{
%       \begin{tikzpicture}[scale=0.2,baseline=-0.5cm]
%           \coordspec
%           \tikzsectorabc[fill=black]{(B)}{(A)}{(C)}{0.3cm}
%           \tikztriangle[yellow][blue][red]{(A)}{(C)}{(D)}
%           \pgfresetboundingbox
%           \path (A) -- (C) -- (D) -- cycle;
%            \draw[red,ultra thick] (E) -- (A) -- (D);
%       \end{tikzpicture}
%   }
%   \newcommand{\DEB}{
%       \begin{tikzpicture}[scale=0.2]
%           \coordspec
%           \tikztriangle[black][blue][yellow]{(D)}{(E)}{(B)}
%           \pgfresetboundingbox
%           \path (D) -- (E) -- (B) -- cycle;
%       \end{tikzpicture}
%   }
%   \newcommand{\EDC}{
%       \begin{tikzpicture}[scale=0.2]
%           \coordspec
%           \tikztriangle[black][blue][yellow]{(E)}{(D)}{(C)}
%           \pgfresetboundingbox
%           \path (E) -- (D) -- (C) -- cycle;
%       \end{tikzpicture}
%   }
%   \newcommand{\bac}{
%     \begin{tikzpicture}
%       \coordspec
%       \tikzsectorabc[fill=black]{(B)}{(A)}{(C)}{0.5cm}
%     \end{tikzpicture}
%   }
%   \newcommand{\abe}{
%     \begin{tikzpicture}
%       \coordspec
%       \tikzsectorabc[fill=red]{(E)}{(B)}{(A)}{0.5cm}
%     \end{tikzpicture}
%   }
%   \newcommand{\acd}{
%     \begin{tikzpicture}
%       \coordspec
%       \tikzsectorabc[fill=red]{(A)}{(C)}{(D)}{0.5cm}
%     \end{tikzpicture}
%   }
%   \newcommand{\adc}{
%     \begin{tikzpicture}
%       \coordspec
%       \tikzsectorabc[fill=blue]{(E)}{(D)}{(A)}{0.5cm}
%       \tikzsectorabc[fill=yellow]{(C)}{(D)}{(E)}{0.5cm}
%     \end{tikzpicture}
%   }
%   \newcommand{\aeb}{
%     \begin{tikzpicture}
%       \coordspec
%       \tikzsectorabc[fill=blue]{(A)}{(E)}{(D)}{0.5cm}
%       \tikzsectorabc[fill=yellow]{(D)}{(E)}{(B)}{0.5cm}
%     \end{tikzpicture}
%   }
%   \newcommand{\ade}{
%     \begin{tikzpicture}
%       \coordspec
%       \tikzsectorabc[fill=blue]{(E)}{(D)}{(A)}{0.5cm}
%     \end{tikzpicture}
%   }
%   \newcommand{\aed}{
%     \begin{tikzpicture}
%       \coordspec
%       \tikzsectorabc[fill=blue]{(A)}{(E)}{(D)}{0.5cm}
%     \end{tikzpicture}
%   }
%   \newcommand{\dec}{
%     \begin{tikzpicture}
%       \coordspec
%       \tikzsectorabc[fill=white,draw=black]{(B)}{(E)}{(C)}{0.5cm}
%       \draw (E) -- ($(E)!0.5cm!(C)$);
%       \tikzsectorabc[fill=yellow]{(D)}{(E)}{(B)}{0.5cm}
%     \end{tikzpicture}
%   }
%   \newcommand{\edc}{
%     \begin{tikzpicture}
%       \coordspec
%       \tikzsectorabc[fill=yellow]{(C)}{(D)}{(E)}{0.5cm}
%     \end{tikzpicture}
%   }
%   \newcommand{\deb}{
%     \begin{tikzpicture}
%       \coordspec
%       \tikzsectorabc[fill=yellow]{(D)}{(E)}{(B)}{0.5cm}
%     \end{tikzpicture}
%   }
%   \newcommand{\edb}{
%     \begin{tikzpicture}
%       \coordspec
%       \tikzsectorabc[fill=white,draw=black]{(B)}{(D)}{(C)}{0.5cm}
%       \draw (D) -- ($(D)!0.5cm!(B)$);
%       \tikzsectorabc[fill=yellow]{(C)}{(D)}{(E)}{0.5cm}
%     \end{tikzpicture}
%   }
%   \renewcommand{\ab}{
%     \tikzhline[red]{0.5cm}\tikzhline[yellow]{0.5cm}
%   }
%   \renewcommand{\ac}{
%     \tikzhline[red]{0.5cm}\tikzhline[yellow]{0.5cm}
%   }
%   \newcommand{\ad}{
%     \tikzhline[red]{1cm}
%   }
%   \renewcommand{\ae}{
%     \tikzhline[red]{1cm}
%   }
%   \newcommand{\db}{
%     \tikzhline[yellow]{1cm}
%   }
%   \newcommand{\dc}{
%     \tikzhline[blue]{1cm}
%   }
%   \newcommand{\eb}{
%     \tikzhline[blue]{1cm}
%   }
%   \newcommand{\ec}{
%     \tikzhline[yellow]{1cm}
%   }
%
%%   
  
 \newcommand{\ABC}{%
   \begin{tikzpicture}[scale=0.5]
       \coordspec
       \tikztriangle[black][blue][red]{(A)}{(B)}{(C)}
       \pgfresetboundingbox
       \path[draw=white,densely dashed,very thick] (B') -- (B);
       \path (A) -- (B) -- (C) -- cycle;
   \end{tikzpicture}
 }
 \newcommand{\ABprimeC}{%
%
 }
 \newcommand{\bac}{%
%
 }
 \newcommand{\bca}{%
%
 }
 \newcommand{\ab }{%
%
 }
 \newcommand{\bc}{%
%
 }
 \newcommand{\ac }{%
%
 }
 \newcommand{\abprime }{%
%
 }
 \newcommand{\bprimec}{%
%
 }
  
  \begin{prop}{\lettrine[lines=2]{I}n}
    any triangle (\ABC) if two angles (\bac and \bca) are equal the 
    sides (\ab and \bc) opposite to them are also equal.
  \end{prop}
  \begin{proof}
    For if the sides be not equal, let one of them \ab be greater than 
    the other \bc, and from it cut off $\abprime \equals \bc$ 
    (\refprop{3}), draw \bprimec. 

    Then in \ABprimeC and \ABC, $\abprime \equals \bc$ 
    (\byconstruction) $\bac \equals \bac$ (\byhypothesis) and \ac 
    common, \therefore the triangles are equal (\refprop{4}) a part 
    equal to the whole, which is absurd; \therefore neither of the 
    sides \ab or \bc is greater than the other, \therefore hence they 
    are equal. 
  \end{proof}
