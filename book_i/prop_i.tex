\pagestyle{euclidprob}
%        \begin{wrapfigure}[3]{r}[0.2\textwidth]{0.6\textwidth}
    \begin{figure}
        \begin{tikzpicture}[scale=1.2]
        \coordinate (A) at (0,0);
        \coordinate (B) at (1,0);
        \draw[name path=C1,blue,ultra thick] (A) circle (1cm);
        \draw[name path=C2,red,ultra thick] (B) circle (1cm);
        \path [name intersections={of=C1 and C2}];
        \coordinate (C) at (intersection-1);
    \def\len{0.55pt}
        \tikztriangle[black][red][yellow]{(A)}{(B)}{(C)}
    \def\len{0.8pt}
        \end{tikzpicture}
    \end{figure}
%        \end{wrapfigure}
    \begin{prop}{\lettrine[lines=2]{\textrm{O}}{\textsc{n}}}
        a given finite straight line 
        (\tikz[baseline=-0.5ex]\draw[ultra thick] (0,0) -- (1,0);) 
        to describe an equilateral triangle.\par
\end{prop}


\begin{proof}
Describe 
\begin{tikzpicture}[scale=0.5,baseline=-0.5ex]
\coordinate (A) at (0,0);
\coordinate (B) at (1,0);
\draw[name path=C1,blue,ultra thick] (A) circle (1cm);
\draw[ultra thick] (A) -- (B); 
\end{tikzpicture}
    and 
\begin{tikzpicture}[scale=0.5,baseline=-0.5ex]
\coordinate (A) at (0,0);
\coordinate (B) at (1,0);
\draw[name path=C2,red,ultra thick] (B) circle (1cm);
\draw[ultra thick] (A) -- (B); 
\end{tikzpicture}
  (postulate \ref{post3}.); draw 
  \tikzhline[yellow]{0.8cm} and
  \tikzhline[red]{0.8cm}
  (\refpost{1}). then will 
  \begin{tikzpicture}[scale=0.5,baseline=-0.5ex]
  \coordinate (A) at (0,0);
  \clip (-0.1,-0.1) rectangle (1.1,1.1);
  \coordinate (B) at (1,0);
  \path[name path=C1,blue,ultra thick] (A) circle (1cm);
  \path[name path=C2,red,ultra thick] (B) circle (1cm);
  \path [name intersections={of=C1 and C2}];
  \coordinate (C) at (intersection-1);
    \def\len{1.6pt}
      \tikztriangle[black][red][yellow]{(A)}{(B)}{(C)}
    \def\len{0.8pt}
  \end{tikzpicture}
be equilateral. 
\begin{align*}
            \text{For }  \tikz[baseline=-0.5ex]\draw[black,ultra thick] (0,0) -- (1,0); 
            &\equals \tikz[baseline=-0.5ex]\draw[yellow,ultra thick] (0,0) -- (1,0);  
            \text{ (\refdef{15})}\\
            \text{and  }\tikz[baseline=-0.5ex]\draw[black,ultra thick] (0,0) -- (1,0); 
            &\equals \tikz[baseline=-0.5ex]\draw[red,ultra thick] (0,0) -- (1,0); 
            \text{ (\refdef{15})} \\
            \therefore \tikz[baseline=-0.5ex]\draw[yellow,ultra thick] (0,0) -- (1,0); 
            &\equals \tikz[baseline=-0.5ex]\draw[red,ultra thick] (0,0) -- (1,0); 
            \text{ (\refax{1})} 
\end{align*}
\end{proof}
