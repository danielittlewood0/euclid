	\pagestyle{euclidthm}
    \begin{figure}[h]
        \begin{tikzpicture}[scale=0.8]
            \node (A) at (0,5)    {A};
            \node (B) at (-3.5,-2){B};
            \node (C) at (3.5,-2) {C};
            \path (A) -- (B) node[pos=0.6] (D) {D};
            \path (A) -- (C) node[pos=0.6] (E) {E};
            \tikzsectorabc[fill=black]{(B)}{(A)}{(C)}{1cm}
            \tikzsectorabc[fill=blue]{(E)}{(D)}{(A)}{1cm}
            \tikzsectorabc[fill=blue]{(A)}{(E)}{(D)}{1cm}
            \tikzsectorabc[fill=yellow]{(C)}{(D)}{(E)}{1cm}
            \tikzsectorabc[fill=yellow]{(D)}{(E)}{(B)}{1cm}
            \tikzsectorabc[fill=red]{(E)}{(B)}{(A)}{1cm}
            \tikzsectorabc[fill=red]{(A)}{(C)}{(D)}{1cm}
            \tikztriangle[yellow][blue][red]{(A)}{(B)}{(E)}
            \tikztriangle[yellow][blue][red]{(A)}{(C)}{(D)}
            \tikztriangle[red][black][red]{(A)}{(D)}{(E)}
        \end{tikzpicture}
    \end{figure}
    \newcommand{\ADE}{
        \begin{tikzpicture}[scale=0.1]
            \clip (-3.0,-0.2) rectangle (3,5.2);
            \node[coordinate] (A) at (0,5)    {A};
            \node[coordinate] (B) at (-3.5,-2){B};
            \node[coordinate] (C) at (3.5,-2) {C};
            \path (A) -- (B) node[pos=0.6,coordinate] (D) {D};
            \path (A) -- (C) node[pos=0.6,coordinate] (E) {E};
            \tikztriangle[red][black][red]{(A)}{(D)}{(E)}
        \end{tikzpicture}
    }
    \newcommand{\ABE}{
        \begin{tikzpicture}[scale=0.3]
            \node (A) at (0,5)    {A};
            \node (B) at (-3.5,-2){B};
            \node (C) at (3.5,-2) {C};
            \path (A) -- (B) node[pos=0.6] (D) {D};
            \path (A) -- (C) node[pos=0.6] (E) {E};
            \tikzsectorabc[fill=black]{(B)}{(A)}{(C)}{0.3cm}
            \tikztriangle[yellow][blue][red]{(A)}{(B)}{(E)}
            \tikztriangle[yellow][blue][red]{(A)}{(C)}{(D)}
        \end{tikzpicture}
    }
    \newcommand{\ACD}{
        \begin{tikzpicture}[scale=0.3]
            \node (A) at (0,5)    {A};
            \node (B) at (-3.5,-2){B};
            \node (C) at (3.5,-2) {C};
            \path (A) -- (B) node[pos=0.6] (D) {D};
            \path (A) -- (C) node[pos=0.6] (E) {E};
            \tikzsectorabc[fill=black]{(B)}{(A)}{(C)}{0.3cm}
            \tikztriangle[yellow][blue][red]{(A)}{(B)}{(E)}
            \tikztriangle[yellow][blue][red]{(A)}{(C)}{(D)}
        \end{tikzpicture}
    }
    \newcommand{\DEB}{
        \begin{tikzpicture}[scale=0.3]
            \node (A) at (0,5)    {A};
            \node (B) at (-3.5,-2){B};
            \node (C) at (3.5,-2) {C};
            \path (A) -- (B) node[pos=0.6] (D) {D};
            \path (A) -- (C) node[pos=0.6] (E) {E};
            \tikztriangle[yellow][blue][red]{(D)}{(E)}{(B)}
        \end{tikzpicture}
    }
    \newcommand{\EDC}{
        \begin{tikzpicture}[scale=0.3]
            \node (A) at (0,5)    {A};
            \node (B) at (-3.5,-2){B};
            \node (C) at (3.5,-2) {C};
            \path (A) -- (B) node[pos=0.6] (D) {D};
            \path (A) -- (C) node[pos=0.6] (E) {E};
            \tikztriangle[yellow][blue][red]{(E)}{(D)}{(C)}
        \end{tikzpicture}
    }
    \newcommand{\bac}{
        bac
    }
    \newcommand{\abe}{
        abe
    }
    \newcommand{\acd}{
        acd
    }
    \newcommand{\adc}{
        adc
    }
    \newcommand{\ade}{
        ade
    }
    \newcommand{\aeb}{
        aeb
    }
    \newcommand{\aed}{
        aed
    }
    \newcommand{\deb}{
        deb
    }
    \newcommand{\dec}{
        dec
    }
    \newcommand{\edb}{
        edb
    }
    \newcommand{\edc}{
        edc
    }
    \renewcommand{\ab}{
        ab
    }
    \renewcommand{\ac}{
        ac
    }
    \newcommand{\ad}{
        ad
    }
    \renewcommand{\ae}{
        ae
    }
    \newcommand{\db}{
        db
    }
    \newcommand{\dc}{
        dc
    }
    \newcommand{\eb}{
        eb
    }
    \newcommand{\ec}{
        ec
    }

    

    \begin{prop}{\lettrine[lines=2]{I}n}
        any isosceles triangle \ADE if the equal sides be produced, the external angles at the base are equal, and the internal angles at the base are also equal. 
	\end{prop}
	\begin{proof}
        Produce \ad, and \ae, (\refpost{2}), take $\db \equals \ec$, (\refprop{3}); draw \dc and \eb. 

        Then in \ABE and \ADE we have, $\ab \equals \ac$ (const.), \bac common to both, and $\ad \equals \ae$ (hyp.) $\therefore \adc \equals \aeb$, $\dc \equals \eb$ and $\abe \equals \acd$ (\refprop{4}). Again in \DEB and \EDC we have $\db \equals \ec$, $\abe \equals \acd$ and $\dc \equals \eb$, \therefore\ $\edb \equals \dec$ and $\edc \equals \deb$  (\refprop{4}) but $\adc \equals \aeb$, \therefore $\ade \equals \aed$ (\refax{3}). 
	\end{proof}
